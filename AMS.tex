% % % % % % % % % % % % % % % % % % % % % % % % % % % % % % % % % % % % % % % %
% LaTeX4EI Template for Cheat Sheets                                Version 1.1
%
% Authors: Markus Hofbauer
% Contact: info@latex4ei.de
% Encode: UTF-8
% % % % % % % % % % % % % % % % % % % % % % % % % % % % % % % % % % % % % % % %


% ======================================================================
% Document Settings
% ======================================================================

% possible options: color/nocolor, english/german, threecolumn
% defaults: color, english, fourcolumn
\documentclass[english]{latex4ei/latex4ei_sheet}

% set document information
\title{Analog and Mixed-Signal Circuit Design}
% \author{LaTeX4EI}                    % optional, delete if unchanged
% \myemail{info@latex4ei.de}           % optional, delete if unchanged
% \mywebsite{www.latex4ei.de}          % optional, delete if unchanged


% ======================================================================
% Begin
% ======================================================================
\begin{document}

% Title
% ----------------------------------------------------------------------
\maketitle   % requires ./img/Logo.pdf


% Section
% ----------------------------------------------------------------------
\section{Semiconductor process and device basics}


    \subsection{MOSFET Basics}
    \begin{center}
        \includegraphics[width=0.58\linewidth]{img/cmos.png}
    \end{center}
    \begin{itemize}
        \item Strong inversion (long-channel):
        \[ I_D = \frac{1}{2}K_n\frac{W}{L}V_{OV}^2(1+\lambda V_{DS}),\quad V_{OV}=V_{GS}-V_{th} \]
        \item Region checks (NMOS):
        \[ \text{ON: }V_{GS}>V_{th},\quad \text{sat.: }V_{DS}\ge V_{OV},\quad \text{triode: }V_{DS}<V_{OV} \]
        \item Small-signal parameters at bias point:
        \[ g_m=\frac{\partial I_D}{\partial V_{GS}}\approx \frac{2I_D}{V_{OV}},\quad g_{ds}\approx \lambda I_D,\quad r_o=\frac{1}{g_{ds}} \]
        \item Also: $g_m = K_n\frac{W}{L}V_{OV} = \sqrt{2K_n\frac{W}{L}I_D}$.
        Weak/mod.\ inv.: $g_m \approx \frac{I_D}{nV_T}$, $n\!\approx\!1\!\ldots\!1.5$, $nV_T\!\approx\!60\,\mathrm{mV}$ at $300\,\mathrm{K}$.
        \item Diode-connected ($V_{DS}\!=\!V_{GS}$): always in sat.\ when ON.
        \item $C_{ox}=\varepsilon_0\varepsilon_{r}/d_{ox}$. NMOS: higher $\mu$. PMOS: lower 1/f noise, no body effect (own n-well), used for low input CM.
    \end{itemize}



    \subsection{MOSFET Advanced (Body Effect, Temp, Reliability)}
    \begin{itemize}
        \item \textbf{Body Effect:} If $V_{BS} \neq 0$, the threshold voltage changes: $V_{th} = V_{th0} + \gamma (\sqrt{|2\Phi_F + V_{SB}|} - \sqrt{|2\Phi_F|})$. Increases $V_{th}$ for NMOS if $V_S > V_B$.
        \item \textbf{Body Transconductance:} $g_{mb} = \eta \cdot g_m$, where $\eta = \gamma / (2\sqrt{|2\Phi_F + V_{SB}|}) \approx 0.1 \dots 0.3$.
        \item \textbf{Temperature Effects:} Mobility $\mu$ decreases with temperature ($\propto T^{-1.5}$). Threshold voltage $V_{th}$ decreases with temperature ($\approx -2$ mV/K).
        \item \textbf{Zero-TC Point:} At a specific $V_{GS}$, the mobility and $V_{th}$ temperature effects cancel out, making $I_D$ temperature-independent.
        \item \textbf{Reliability (HCI):} Hot Carrier Injection. High electric field near the drain accelerates carriers, which can get trapped in the oxide, shifting $V_{th}$ over time.
        \item \textbf{Reliability (NBTI):} Negative Bias Temperature Instability. Affects PMOS under negative gate bias at high temperatures, increasing $|V_{th}|$.
    \end{itemize}

    \subsection{Basic MOS Stages (CS/CD/CG)}
    \begin{center}
        \includegraphics[width=0.92\linewidth]{img/stages.png}
    \end{center}
    \begin{center}
        \includegraphics[width=0.5\linewidth]{img/cs_ss.png}
    \end{center}
    \begin{center}
        \includegraphics[width=0.5\linewidth]{img/cd_ss.png}
    \end{center}
    \begin{center}
        \includegraphics[width=0.5\linewidth]{img/cg_ss.png}
    \end{center}
    Assume saturation and finite $r_o$.
    \begin{itemize}
        \item Common-Source (load $R$):
        \[ A_V\approx -g_m(R\parallel r_o),\quad r_{in}\approx \infty,\quad r_{out}\approx R\parallel r_o \]
        \item Common-Drain / Source Follower:
        \[ A_V\approx \frac{g_m(R\parallel r_o)}{1+g_m(R\parallel r_o)}\approx 1,\quad r_{in}\approx \infty \]
        \[ r_{out}\approx \frac{R\parallel r_o}{1+g_m(R\parallel r_o)}\approx \frac{1}{g_m}\ (g_mR\gg1) \]
        \item Common-Gate:
        \[ A_V\approx \frac{g_mR+R/r_o}{1+R/r_o}\approx g_mR\ (r_o\gg R) \]
        \[ r_{in}\approx \frac{R+r_o}{1+g_mr_o}\approx \frac{1}{g_m},\quad r_{out}\approx R\parallel r_o \]
        \item CS large-signal transfer (ignore CLM):
        \[ V_{out}=V_{DD}-R\cdot \frac{1}{2}K_n\frac{W}{L}(V_{in}-V_{th})^2,\quad V_{in,min}=V_{th} \]
        \item CS input voltage range for saturation:
        \[ V_{in,max} = -\frac{1}{R K_n \frac{W}{L}} + \sqrt{\left(\frac{1}{R K_n \frac{W}{L}}\right)^2 + \frac{2 V_{DD}}{R K_n \frac{W}{L}}} + V_{th} \]
    \end{itemize}


    \subsection{Current Mirrors (Basic \& Cascode)}
    \begin{center}
        \includegraphics[width=0.3\linewidth]{img/cm.png}
    \end{center}
    \begin{center}
        \includegraphics[width=0.92\linewidth]{img/cm_ss.png}
    \end{center}
    \begin{center}
        \includegraphics[width=0.6\linewidth]{img/ccm_ss.png}
    \end{center}
    \begin{itemize}
        \item Basic mirror (sat., ignore CLM):
        \[ I_{out}=I_{in}\cdot \frac{(W/L)_2}{(W/L)_1} \]
        \item Output compliance (PMOS mirror example):
        \[ V_{SD,2}>V_{OV,2}\ \Rightarrow\ V_{out}<V_{DD}-V_{OV,2} \]
        \item Output compliance (NMOS mirror example):
        \[ V_{DS,2}>V_{OV,2}\ \Rightarrow\ V_{out}>V_{OV,2} \]
        \item With CLM, ratio depends on output voltage:
        \[ \frac{I_{out}}{I_{in}}=\frac{(W/L)_2}{(W/L)_1}\cdot\frac{1+\lambda V_{out}}{1+\lambda V_{DS,1}} \]
        \item Standard mirror output resistance:
        \[ r_{out}\approx r_{o,2} \]
        \item Cascode mirror: drain of mirror device is shielded from $V_{out}$ changes
        \[ r_{out,cas}\approx g_{m,cas}\,r_{o,cas}\,r_{o,2}\gg r_o \]
        \item Trade-off: much better accuracy/output resistance, but higher minimum output voltage (reduced swing).
        \item Current transfer function (small signal):
        \[ \frac{i_{out}}{i_{in}} = \frac{g_{m2}}{g_{m1} + g_{ds1}} \]
        \item Cascode bias: $V_b = V_{GS,1}+V_{GS,4}=2V_{GS}$ (matched).
        \item Cascode $V_{out,min}=2V_{DS,sat}+V_{th}$ (standard); $2V_{DS,sat}$ (wide-swing).
    \end{itemize}


    \subsection{Resistive/Capacitive Loading \& Corners}
    CS with output load $R\parallel 1/(sC)$:
    \begin{itemize}
        \item Transfer:
        \[ A_V(s)=\frac{V_{out}}{V_{in}}=-\frac{g_m}{1/R+sC}=-\frac{g_mR}{1+sRC} \]
        \item Low-frequency gain and pole:
        \[ |A_0|=g_mR,\quad \omega_p=\frac{1}{RC},\quad f_p=\frac{1}{2\pi RC} \]
        \item At the pole:
        \[ |A_V(\omega_p)|=\frac{|A_0|}{\sqrt{2}} \]
        \item Resistor PVT model:
        \[ R(T)=R_{nom}\big(1+\alpha(T-T_0)\big),\quad \alpha\,[\mathrm{ppm}/^\circ\mathrm{C}] \]
        \item Corner intuition: larger $R\Rightarrow$ larger gain, lower pole frequency.
    \end{itemize}


    \subsection{Noise, Mismatch, and Layout-Relevant Sizing}
    \begin{center}
        \includegraphics[width=0.92\linewidth]{img/noise.png}
    \end{center}
    \begin{itemize}
        \item Thermal noise PSD:
        \[ \overline{v_n^2}=4kTR\ [\mathrm{V}^2/\mathrm{Hz}],\quad \overline{i_{n,th}^2}=4kT\gamma g_m\ [\mathrm{A}^2/\mathrm{Hz}] \]
        \item Flicker noise (MOS gate-referred):
        \[ \overline{v_{n,1/f}^2}=\frac{K_f}{C_{ox}WL\,f},\quad \overline{i_{n,1/f}^2}=g_m^2\overline{v_{n,1/f}^2} \]
        \item 1/f corner from $\overline{i_{n,th}^2}=\overline{i_{n,1/f}^2}$:
        \[ f_c=\frac{K_f g_m}{4kT\gamma C_{ox}WL} \]
        \item Input-referred noise of CS with active load (thermal):
        \[ \overline{v_{n,in}^2}=4kT\gamma\left(\frac{1}{g_{m1}}+\frac{g_{m2}}{g_{m1}^2}\right) \]
        \item Input-referred concept (generic):
        \[ \overline{v_{n,in}^2}=\frac{\overline{v_{n,out}^2}}{|A_v|^2},\quad \overline{i_{n,in}^2}=\frac{\overline{i_{n,out}^2}}{|A_i|^2} \]
        \item Integrate PSD over band $[f_1,f_2]$:
        \[ \overline{x_n^2}=\int_{f_1}^{f_2}S_x(f)\,df,\quad \int_{f_1}^{f_2}\frac{1}{f}df=\ln\!\frac{f_2}{f_1} \]
        \item Noise design rule: maximize signal-device $g_{m1}$ (large $W/L$), minimize load-device $g_{m2}$ (small $W/L$) under power/headroom constraints.
        \item Capacitor mismatch (1$\sigma$):
        \[ \sigma_C/C\approx \frac{K_C}{\sqrt{WL}} \]
        \item For switched-cap top-plate equations, mismatch in one branch appears as gain/threshold error; increase capacitor area to reduce error.
        \item PMOS transistors show less 1/f noise.
    \end{itemize}


\section{Amplifiers}

    \subsection{Ideal \& Real Op-Amps (Feedback)}
    \begin{itemize}
        \item \textbf{Ideal Op-Amp:} $A_v \to \infty$, $R_{in} \to \infty$, $R_{out} \to 0$, infinite bandwidth.
        \item \textbf{Feedback Equation:} Closed-loop gain $A_{CL} = \frac{A}{1 + A\beta}$, where $A$ is open-loop gain and $\beta$ is the feedback factor.
        \item \textbf{Loop Gain:} $T = A\beta$. Determines accuracy, bandwidth extension, and impedance modification.
        \item \textbf{Gain Error:} $\epsilon \approx \frac{1}{A\beta}$ (for large $A\beta$).
        \item \textbf{Bandwidth:} Closed-loop bandwidth $\omega_{CL} = \omega_{OL} \cdot (1 + A_0\beta) \approx \beta \cdot \text{GBW}$.
        \item \textbf{Impedances:} Series feedback increases impedance by $(1+A\beta)$. Shunt feedback decreases impedance by $(1+A\beta)$.
        \item \textbf{Non-Idealities:} Finite gain, finite bandwidth, offset voltage ($V_{OS}$), input bias currents, finite slew rate, noise.
    \end{itemize}

    \subsection{Differential Pair \& OTA (Exam Core)}
    Symmetric differential excitation: $v_{in1}=+v_{id}/2,\ v_{in2}=-v_{id}/2$.
    \begin{itemize}
        \item Half-circuit rule (ideal tail source): source node is small-signal virtual ground.
        \item Resistive-load differential pair:
        \[ v_{od}=v_{out1}-v_{out2}\approx -g_mR_D\,v_{id},\quad A_{vd}\approx -g_mR_D \]
        \item Single-ended output from one branch:
        \[ A_v\approx -\frac{1}{2}g_mR_D \]
        \item Differential pair with active load (OTA, rough):
        \[ A_v\approx g_m\,r_{out},\quad r_{out}\approx r_{o,n}\parallel r_{o,p}\parallel R_L \]
        \item Common-mode gain and rejection:
        \[ A_{cm}=\frac{v_{od}}{v_{icm}},\quad \mathrm{CMRR}=\left|\frac{A_{vd}}{A_{cm}}\right|,\quad \mathrm{CMRR}_{dB}=20\log_{10}(\mathrm{CMRR}) \]
        \item Input common-mode range (resistive-load pair):
        \[ V_{in,CM,min}\approx V_{GS}+V_{DS,sat,tail},\quad V_{in,CM,max}\approx V_{out,CM}-V_{DS,sat}-V_{GS} \]
        \item Output common-mode with resistive loads:
        \[ V_{out,CM}\approx V_{DD}-\frac{I_{SS}}{2}R_D \]
        \item CM$\to$DM gain (finite tail $R_S$, $g_m$ mismatch $\Delta g_m$):
        \[ A_{CM\to DM}=\frac{\Delta g_m\, R}{1+(g_{m1}+g_{m2})R_S} \]
        \item $\mathrm{CMRR}=\left|\frac{A_{vd}}{A_{CM\to DM}}\right|$;\quad $\mathrm{PSRR}=\left|\frac{A_v}{A_{v,\mathrm{supply}}}\right|$.
        \item Input-referred offset ($\Delta V_{th}$, $\Delta K$ mismatch):
        \[ V_{OS}\approx \Delta V_{th} - \frac{\Delta K}{2K}V_{OV} = \Delta V_{th} - \frac{\Delta K}{K}\frac{I_D}{g_m} \]
        Low offset: large $g_m/I_D$ (large $W/L$) for input pair, small $g_m/I_D$ for mirrors.
    \end{itemize}

    \subsection{Output Stages (Class A/B, Quick Checks)}
    \begin{itemize}
        \item Class-A common-source output stage:
        \[ r_{out}\approx \frac{1}{g_{ds,n}+g_{ds,p}}\approx \frac{1}{(\lambda_n+\lambda_p)I_B} \]
        \item To reduce $r_{out}$: increase $I_B$ (trade-off: static power $P_Q\approx V_{DD}I_B$).
        \item Resistive load gain limit:
        \[ A_v\approx g_m\left(r_{o,n}\parallel r_{o,p}\parallel R_L\right) \]
        \item Class-B push-pull: each device conducts about half-cycle; near zero crossing crossover distortion appears.
        \item Class-AB biasing reduces crossover distortion at small static current penalty.
    \end{itemize}

    \subsection{2nd-Order Amplifier Filters (Biquad LP)}
    \begin{itemize}
        \item Canonical low-pass biquad:
        \[ H(s)=\frac{\omega_n^2}{s^2+\frac{\omega_n}{Q}s+\omega_n^2} \]
        \item Pole locations:
        \[ s_{p1,2}=-\frac{\omega_n}{2Q}\pm\sqrt{\left(\frac{\omega_n}{2Q}\right)^2-\omega_n^2} \]
        \item Complex poles iff $Q>1/2$; at $Q=1/2$: double real pole at $-\omega_n$.
        \item Magnitude squared:
        \[ |H(j\omega)|^2=\frac{\omega_n^4}{(\omega_n^2-\omega^2)^2+\left(\frac{\omega_n}{Q}\omega\right)^2} \]
        \item Peaking frequency (if it exists):
        \[ \omega_a=\omega_n\sqrt{1-\frac{1}{2Q^2}},\quad \text{requires }Q>\frac{1}{\sqrt{2}} \]
        \item Trade-off: increasing quality factor gives a steeper transition near the natural frequency, but also more peaking/ringing.
    \end{itemize}


    \subsection{Tow-Thomas Biquad with OpAmps}
    \begin{itemize}
        \item Two ideal integrators alone give:
        \[ H(s)=\frac{k_1k_2}{s^2+k_1k_2} \]
        (equivalent to $Q\to\infty$, undamped).
        \item Lossy first integrator (feedback factor $\alpha$):
        \[ H_1(s)=\frac{k_1}{s+\alpha k_1} \]
        \item Resulting biquad:
        \[ H(s)=\frac{k_1k_2}{s^2+\alpha k_1 s+k_1k_2},\quad \omega_n=\sqrt{k_1k_2},\quad Q=\frac{\sqrt{k_1k_2}}{\alpha k_1} \]
        \item Inverting opamp integrator:
        \[ \frac{V_{out}}{V_{in}}=-\frac{1}{R_1C_1s}\Rightarrow k_1=\frac{1}{R_1C_1} \]
        \item Lossy integrator ($R_2\parallel C_1$ feedback):
        \[ \frac{V_{out}}{V_{in}}=-\frac{1/(R_1C_1)}{s+1/(R_2C_1)},\quad \alpha=\frac{R_1}{R_2} \]
        \item Full Tow-Thomas biquad (with feedback $R_F$):
        \[ \omega_n\!=\!\frac{1}{\sqrt{R_3R_FC_1C_2}},\; Q\!=\!R_2\!\sqrt{\frac{C_1}{R_3R_FC_2}},\; G\!=\!-\frac{R_F}{R_1} \]
        $Q$ and gain $G$ are independently tunable via $R_2$ and $R_1$.
    \end{itemize}


    \subsection{Two-Stage OpAmp \& Miller Compensation}
    Uncompensated 2-stage model:
    \[ A_v(s)=\frac{g_{mI}g_{mII}R_1R_L'}{(1+sR_1C_1)(1+sR_L'C_L')} \]
    \begin{itemize}
        \item DC gain and poles:
        \[ A_0=g_{mI}g_{mII}R_1R_L',\quad p_1\approx-\frac{1}{R_1C_1},\quad p_2\approx-\frac{1}{R_L'C_L'} \]
        \item With Miller capacitor $C_M$ (pole splitting):
        \[ p_1'\approx-\frac{1}{g_{mII}R_1R_L'C_M},\quad p_2'\approx-\frac{g_{mII}}{C_1+C_L'} \]
        \[ z_{RHP}\approx+\frac{g_{mII}}{C_M} \]
        \item Gain-bandwidth product:
        \[ \mathrm{GBW}=|A_0p_1'|\approx\frac{g_{mI}}{C_M} \]
        \item Rule for ~60$^\circ$ phase margin (with RHP zero considered):
        \[ p_2'\approx(2\ldots2.5)\,\mathrm{GBW},\quad C_M\approx2.2\frac{g_{mI}}{g_{mII}}C_L \]
        \item Practical choice: $g_{mII}\approx10\,g_{mI}$ (good pole split with moderate current penalty).
        \item Inverting OpAmp with $R_1, R_2, C_1$:
        \[ \frac{V_{out}}{V_{in}} = -\frac{R_2}{R_1 \cdot (1 + j\omega C_1 R_2)} \]
        \item Stability: Amplifier is stable if Phase Margin $> 0^\circ$.
    \end{itemize}


    \subsection{Miller OpAmp Design Flow (Exam Steps)}
    \begin{itemize}
        \item Start from phase-margin target: choose $g_{mII}/g_{mI}$ and compute $C_M$.
        \item From transit frequency $f_T$:
        \[ g_{mI}=2\pi f_T C_M \]
        \item From slew-rate target:
        \[ SR\approx\frac{I_{tail}}{C_M}\Rightarrow I_{tail}\ge SR\cdot C_M \]
        \item Input pair (noise-efficient): often moderate inversion ($V_{OV}\approx60\,\mathrm{mV}$);
        current sources/output stage: strong inversion ($V_{OV}\approx200\,\mathrm{mV}$).
        \item Stage gains (rough):
        \[ A_{v1}\approx\frac{g_{m1}}{g_{ds,\text{loads}}},\quad A_{v2}\approx\frac{g_{m6}}{g_{ds,6}+g_{ds,7}+1/R_L},\quad A_0\approx A_{v1}A_{v2} \]
        \item Avoid systematic offset in mirrors: keep mirrored devices at equal $V_{DS}$ at operating point.
        \item Nulling resistor to cancel Miller zero:
        \[ R_z\approx\frac{1}{g_{mII}} \]
        \item Folded cascode: NMOS mirror has higher bandwidth (higher mobility $\rightarrow$ smaller devices $\rightarrow$ less cap). PMOS mirror has lower noise (less current $\rightarrow$ less $g_m$ in current source $\rightarrow$ less noise).
    \end{itemize}


    \subsection{Advanced Gain Stages}
    \begin{itemize}
        \item \textbf{Gain Boosting:} Uses an auxiliary amplifier to increase the output impedance of a cascode stage without reducing voltage headroom. $A_{boosted} \approx A_{cascode} \cdot A_{aux}$.
        \item \textbf{Pole-Zero Doublet:} Gain boosting introduces a pole-zero doublet. If the doublet is below the unity-gain frequency, it degrades settling time. Ensure $\omega_{p,aux} > \beta \cdot \omega_{u,main}$.
        \item \textbf{Offset Reduction (Chopping):} Modulates the input signal to a higher frequency, amplifies it, and demodulates it back to baseband. 1/f noise and DC offset are modulated away from baseband.
        \item \textbf{Offset Reduction (Auto-Zeroing):} Samples the offset during a reset phase and subtracts it during the amplification phase. Increases baseband noise due to noise aliasing (folding).
    \end{itemize}

\section{References \& comparators}

    \subsection{Bandgap References (Low-Voltage)}
    \begin{itemize}
        \item Principle:
        \[ V_{ref}=V_{BG}=V_{BE}+K\,V_T \]
        with $V_{BE}$ CTAT and $V_T$ PTAT.
        \item Temperature compensation condition:
        \[ \frac{dV_{BG}}{dT}=0\Rightarrow K=-\frac{dV_{BE}/dT}{dV_T/dT} \]
        (given exercise values: $K\approx17.2$).
        \item Using emitter-area ratio $n$:
        \[ \Delta V_{BE}=V_T\ln(n) \]
        \item Low-voltage topology expression:
        \[ V_{BG}=\frac{R_4}{R_2}\left(V_{BE1}+\frac{R_2}{R_1}V_T\ln(n)\right) \]
        \item Design knobs:
        $\frac{R_2}{R_1}\ln(n)$ sets tempco cancellation, $\frac{R_4}{R_2}$ sets output level.
        \item Minimum supply (small current, large devices):
        \[ V_{DD,min}\approx V_{BE}+V_{DS,sat} \]
        \item Startup: add startup device to pull internal node out of zero-current metastable state during ramp-up.
        \item Bipolar transistors are most commonly used in bandgap/voltage reference circuits.
    \end{itemize}


    \subsection{Linear Voltage Regulator (LDO)}
    \begin{itemize}
        \item \textbf{Basics:} Uses an error amplifier to compare a fraction of the output voltage with a reference voltage ($V_{ref}$) and drives a pass transistor to regulate $V_{out}$.
        \item \textbf{Output Voltage:} $V_{out} = V_{ref} \cdot (1 + R_1/R_2)$, where $R_1, R_2$ form the feedback divider.
        \item \textbf{Dropout Voltage:} Minimum $V_{DS}$ of the pass transistor required to maintain regulation. PMOS pass devices allow low dropout ($V_{DS,sat}$).
        \item \textbf{Pass-device sizing at }$I_{out,max}$ (PMOS, sat. model):
        \[ \frac{W}{L}\ge \frac{2I_{out,max}}{k_pV_{OV}^2} \]
        \item \textbf{Load limit from current limit:}
        \[ R_{L,min}=\frac{V_{out}}{I_{out,max}} \]
        \item \textbf{Stability:} The pass transistor and load capacitor form a low-frequency pole. The error amplifier forms another pole. Requires compensation (e.g., ESR of the load capacitor or internal Miller compensation) to ensure Phase Margin $> 0^\circ$.
        \item \textbf{PSRR:} Power Supply Rejection Ratio depends on the loop gain. High loop gain at low frequencies provides good PSRR.
        \item \textbf{Efficiency (ideal, neglect }$I_Q$\textbf{):}
        \[ \eta\approx\frac{P_{out}}{P_{in}}\approx\frac{V_{out}}{V_{in}} \]
    \end{itemize}

    \subsection{Switches \& Switched-Capacitor Basics}
    \begin{itemize}
        \item Continuous-time inverting amplifier with finite $A_v,\,R_{out}$ shows closed-loop gain error (vs. ideal $-R_2/R_1$): finite output resistance and finite gain both degrade accuracy.
        \item High-gain approximation (if $A_v\gg1$ and $R_{out}\gg R_2$):
        \[ \frac{V_{out}}{V_{in}}\approx-\frac{R_2}{R_1}\left(1-\frac{1+R_2/R_1}{A_v}\right) \]
        \item SC charge-transfer concept: each clock cycle moves a packet $\Delta Q=C\,\Delta V$.
        \item Equivalent-resistor intuition:
        \[ I_{avg}=f_s\,C\,(V_1-V_2),\quad R_{eq}=\frac{1}{C f_s} \]
        \item Two-capacitor sharing ($C_2$ initially uncharged):
        \[ V_{final}=\frac{C_1}{C_1+C_2}V_{in} \]
        \item For $C_2\gg C_1$, per-cycle output step is small (better linear integration behavior, easier headroom control).
        \item Main switch non-idealities:
        on-resistance/settling, limited input range, aperture uncertainty, charge injection, clock feedthrough, nonlinear junction caps.
    \end{itemize}


    \subsection{SC Integrator (Discrete-Time Core)}
    \begin{itemize}
        \item Ideal charge transfer each integration phase:
        \[ C_1V_{in}[k]\rightarrow C_2\Delta V_{out}[k] \]
        \item Difference equation (inverting):
        \[ V_{out}[k]=V_{out}[k-1]-\frac{C_1}{C_2}V_{in}[k-1] \]
        \item Constant input $V_{in}$ gives ramp:
        \[ V_{out}[k]=-k\frac{C_1}{C_2}V_{in} \]
        \item Unlike direct capacitor sharing, integration linearity does not require $C_1\ll C_2$; ratio mainly sets step size/dynamic range.
        \item Parasitic-insensitive implementation:
        use bottom-plate sampling and clock phasing so major parasitic/charge-injection errors become constant or cancelable.
    \end{itemize}


    \subsection{Comparators}
    \begin{itemize}
        \item \textbf{Concept:} High-gain amplifier without compensation (open-loop) to quickly resolve small input differences to logic levels.
        \item \textbf{Hysteresis:} Positive feedback creates two different threshold voltages ($V_{TH,H}$ and $V_{TH,L}$) to prevent noise-induced multiple transitions.
        \item \textbf{Clocked Comparators (Latches):} Use positive feedback (cross-coupled inverters) activated by a clock to quickly regenerate small signals to full logic levels. Saves static power.
        \item \textbf{Offset Compensation:} Auto-zeroing techniques store the input offset voltage on a capacitor during a reset phase and subtract it during the evaluation phase.
        \item \textbf{Regeneration time (latch intuition):}
        \[ t_{reg}\propto \tau\ln\!\left(\frac{V_{logic}}{\Delta V_{in}}\right) \]
        Smaller input overdrive $\Delta V_{in}$ increases delay and decision jitter.
        \item Reference noise/ripple must stay well below 1\,LSB; improve with loop gain (in-band PSRR) and bypass caps.
    \end{itemize}



% ======================================================================
% End
% ======================================================================
\end{document}
